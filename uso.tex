\begin{enumerate}
\item El cliente realiza un pedido online. Acuerda la fecha de entrega. Paga online. La página notifica el pago. El encargado de armar los pedidos lo arma. Llegada la fecha de entrega, el encargado de entrega envia la notificación de que el pedido esta en camino. El cliente recibe la notificación de que el pedido esta en camino. Se encuentra en la casa en el momento en el que arriba la entrega. Se realiza la entrega. El encargado de entrega notifica la entrega.
\item El cliente realiza un pedido online. Acuerda la fecha de entrega y elegir pagar contra entrega. El encargado de pedidos notifica que el pedido esta en camino. El pedido es enviado pero el cliente no se encuentra en su domicilio cuando el encargado de las entregas llega. El pedido vuelve al deposito. El cliente es penalizado.
\item El cliente intenta realizar un pedido online. La página lo rechaza por estar penalizado.
\item Un local cuenta con poco stock de un producto. Realiza un pedido a través de la página web. El encargado de armar los pedidos arma el pedido del local. El encargado de entregas tiene un momento libre en el horario en el que el local se encuentra abierto. El encargado de entregas entrega el pedido al local. El encargado de entregas notifica que el pedido fue entregado.
\item El cliente se registra en la página ingresando sus datos personales y una foto sosteniendo su DNI para verificar su identidad. La identidad del cliente es verificada. El cliente logra registrarse exitosamente y ahora cuenta con un usuario para realizar pedidos. 
\item El encargado del stock del depósito recibe una alerta desde la página web informando que el stock de un producto esta por agotarse. Éste se pone en contacto con el proveedor, realiza el pedido del producto en cuestión y acuerda una fecha de entrega. El proveedor enviá los productos al deposito que los solicitó en la fecha acordada y este los recibe, reponiendo así el stock del producto.
\item Se construye un nuevo depósito. El depósito está listo para empezar a enviar pedidos. El gerente agrega el nuevo depósito al sistema. El encargado de stock del depósito configura los umbrales mínimos de cada producto que contiene el nuevo depósito.
\end{enumerate}
