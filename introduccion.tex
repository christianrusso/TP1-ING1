El objetivo del presente trabajo práctico es mostrar, mediante el uso de diferentes técnicas, diagramas y modelos, un proyecto para incrementar las ventas de la cadena de tiendas \textbf{Mes\%}.

Para lograr este objetivo utilizamos dos tipos diferente de diagramas:
\begin{itemize}
	\item\textbf{ Diagramas de contexto}: Este tipo de diagramas permite mostrar claramente los limites del sistema, las interacciones que este tiene con los agentes \textit{(humanos, dispositivos o software)} y las interacciones entre los diferentes agentes.
	\item \textbf{Diagrama de objetivos}: Nos permite mostrar los objetivos que el sistema desea alcanzar y que objetivos contribuyen a que otros objetivos mas generales se cumplan. También nos permite asegurarnos de que comprendemos correctamente que es lo que se desea hacer u obtener, quien estará a cargo de satisfacer cada objetivo, que situaciones o comportamientos asumimos que ocurrirán, presentar diferentes alternativas para cumplir ciertos objetivos, etc.
\end{itemize}

Mediante el uso de los diagramas explicaremos la solución que propusimos para lograr que la cadena \textbf{Mes\%} alcance los siguientes objetivos:
\begin{itemize}
	\item Reducir las colas en los locales.
	\item Mejorar el manejo del stock de productos en los depósitos y en los locales.
	\item Permitir a los locales solicitar stock a los depósitos de forma online.
	\item Proveer a la gerencia estadísticas relevantes sobre las ventas realizadas.
\end{itemize}
